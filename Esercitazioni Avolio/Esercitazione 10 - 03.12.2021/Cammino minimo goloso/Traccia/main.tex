\documentclass{article}
\usepackage[utf8]{inputenc}
\usepackage{graphicx}
\usepackage{float}
\title{Esercitazione - Cammino minimo con tecnica golosa}
\author{Fondamenti 2 - Corso di Laurea in Informatica}
\date{  }

\begin{document}
\maketitle
\noindent Si scriva un programma C++ che dato un grafo pesato $g$ e due interi $i$ e $j$ calcoli, seguendo un approccio di tipo \textit{goloso}, un possibile cammino dal nodo $i$ al nodo $j$ e lo stampi insieme al relativo costo.\\
In seguito, definire un'istanza del problema (grafo, nodo $i$ e nodo $j$) in cui il cammino calcolato con l'approccio goloso definito al punto precedente non coincide con il cammino di costo minimo tra i due nodi.
\end{document}
